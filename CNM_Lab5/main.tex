\documentclass[12,a4paper]{article}

\usepackage{graphicx}
\usepackage{float}
\usepackage{caption}
\usepackage{subcaption}
\usepackage{multirow}
\usepackage{amsmath}
\usepackage{listings}
\usepackage{color}
\usepackage{blindtext}
\definecolor{dkgreen}{rgb}{0,0.6,0}
\definecolor{gray}{rgb}{0.5,0.5,0.5}
\definecolor{mauve}{rgb}{0.58,0,0.82}

\lstset{
  language=Python,
  aboveskip=3mm,
  belowskip=3mm,
  showstringspaces=false,
  columns=flexible,
  basicstyle={\small\ttfamily},
  numbers=none,
  numberstyle=\tiny\color{gray},
  keywordstyle=\color{blue},
  commentstyle=\color{dkgreen},
  stringstyle=\color{mauve},
  breaklines=true,
  breakatwhitespace=true,
  tabsize=3
}

\title{Computational and Numerical Methods}
\date{\today}
\author{Amarnath Karthi  201501005 \\ Chahak Mehta  201501422}

\setlength{\parindent}{0em}

\makeatletter
\begin{document}
    \begin{titlepage}
	    \centering
	    {\scshape\LARGE SC-374 \par}
	    \vspace{0.1cm}
	    {\huge \@title \par}
	    \vspace{0.5cm}
	    {\Large Assignment 5\par}
	    \vspace{10cm}
	    \Large Amarnath Karthi          201501005\\
	    \Large Chahak Mehta             201501422\\
	    \vspace{5cm}
	    {\large \@date\par}
    \end{titlepage}
    
    \section{Accretion discs}
    
    \begin{figure}[H]
        \centering
        \includegraphics[width = \textwidth]{plots/accretion.png}
        \caption{Assignment 2}
        \label{fig:acc_disc}
    \end{figure}
    
    \section{Gaussian Elimination Method}
    \begin{enumerate}
        \item Solve system of linear equations:
            \begin{eqnarray*}
                x_1 + 2x_2 + x_3 &=& 0 \\
                2x_1 + 2x_2 + 3x_3 &=& 3 \\
                -x_1 - 3x_2 &=& 2
            \end{eqnarray*}
            \textbf{Soln:}\\
            \begin{table}[H]
                \centering
                \begin{tabular}{|c|c|}
                    \hline
                    $x_1$ & 1 \\
                    $x_2$ & -1 \\
                    $x_3$ & 1 \\
                    \hline 
                \end{tabular}
            \end{table}

        \item Solve system of linear equations:
            \begin{eqnarray*}
                4x_1 + 3x_2 + 2x_3 + x_4 &=& 1 \\
                3x_1 + 4x_2 + 3x_3 + 2x_4 &=& 1 \\
                2x_1 + 3x_2 + 4x_3 + 3x_4 &=& -1 \\
                x_1 + 2x_2 + 3x_3 + 4x_4 &=& -1 
            \end{eqnarray*}
            \textbf{Soln:}\\
            \begin{table}[H]
                \centering
                \begin{tabular}{|c|c|}
                    \hline
                    $x_1$ &  0\\
                    $x_2$ &  1\\
                    $x_3$ &  -1\\
                    $x_4$ & 0\\ 
                    \hline
                \end{tabular}
            \end{table}

        \item Find the inverse of the following matrix:\\
            \[
            \begin{bmatrix}
                1 & 1 & -1 \\
                1 & 2 & -2 \\
                -2 & 1 & 1
            \end{bmatrix}
            \]
            \textbf{Soln:}\\
            \[
                \begin{bmatrix}
                2.00000000e+00 & -1.00000000e+00 & -2.49800181e-16 \\
                1.50000000e+00 & -5.00000000e-01 &  5.00000000e-01 \\
                2.50000000e+00 & -1.50000000e+00 &  5.00000000e-01
                \end{bmatrix}
            \]
    \end{enumerate}
    
    \section{Jacobi Iteration and Gauss-Seidel Methods}
    \begin{eqnarray*}
        9x_1 + x_2 + x_3 &=& 10 \\
        2x_1 + 10x_2 + 3x_3 &=& 19 \\
        3x_1 + 4x_2 + 11x_3 &=& 0
    \end{eqnarray*}
    Initial guess values of $x^{(0)}_1 = x^{(0)}_2 = x^{(0)}_3 = 0$ and using a tolerance of $0.001$.
    \begin{table}[H]
        \centering
        \begin{tabular}{|c|c|c|c|}
            \hline
            variable & Analytic solution & Jacobi & Gauss-Seidel \\
            $x_1$ & 1 & 1.00000701759 & 1.00000701759 \\
            $x_2$ & 2 & 2.00010775418 & 2.00001652  \\
            $x_3$ & -1 & -0.99985948542 & -1.00000792175 \\
            \hline 
        \end{tabular}
    \end{table}
    \begin{figure}[H]
        \centering
        \begin{subfigure}[t]{0.49\textwidth}
            \includegraphics[width=\textwidth]{plots/Convergence_of_x0.png}
            \caption{Convergence of $x_1$}
            \label{fig:conv_x1}
        \end{subfigure}
        \begin{subfigure}[t]{0.49\textwidth}
            \includegraphics[width=\textwidth]{plots/Convergence_of_x1.png}
            \caption{Convergence of $x_2$}
            \label{fig:conv_x2}
        \end{subfigure}
        \begin{subfigure}[t]{0.49\textwidth}
            \includegraphics[width=\textwidth]{plots/Convergence_of_x2.png}
            \caption{Convergence of $x_3$}
            \label{fig:conv_x3}
        \end{subfigure}
        \caption{A comparison of the convergence of all roots for Jacobi and Gauss-Seidel methods}
    \end{figure}
    
    
    \newpage
    \section{Non-linear Systems using Netwon Raphson method}
    
    \begin{eqnarray*}
        f(x,y) \equiv x^2 + 4y^2 - 9 &=& 0 \\
        g(x,y) \equiv 18y - 14x^2 + 45 &=& 0 \\
    \end{eqnarray*}
    
    \begin{figure}[H]
        \centering
        \includegraphics[width = \textwidth]{plots/non-linear.png}
    \end{figure}
    
    \begin{table}[H]
        \centering
        \begin{tabular}{|c|c|}
        \hline
            $(x_1,y_1)$ & (-2.13721674 , 1.05265196) \\
            $(x_2,y_2)$ & (-1.20316696 , -1.37408053) \\
            $(x_3,y_3)$ & ( 1.20316696 , -1.37408053) \\
            $(x_4,y_4)$ & ( 2.13721674 , 1.05265196) \\
        \hline
        \end{tabular}
    \end{table}
    
    $(x_1,y_1), (x_2,y_2), (x_3,y_3), (x_4,y_4)$ took 4, 5, 5, 4 iterations respectively to converge.
    
\end{document}